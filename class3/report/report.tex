\documentclass[11pt, draftclsnofoot, onecolumn]{IEEEtran} 

\usepackage{epsfig}
\usepackage{times}
\usepackage{subfigure}
\usepackage{amsfonts}
\usepackage{amsmath}
\usepackage{amssymb}
\usepackage{graphicx}
\usepackage{url}
%\usepackage{cite}
\usepackage{psfrag}
%\usepackage{stfloats}
\usepackage{array}
\usepackage{mdwmath} % part of Mark Wooding's powerful
\usepackage{mdwtab}  % tools for math, tables, ...
\usepackage{multirow}
\usepackage{verbatim}
%\usepackage{booktabs}
%\usepackage{rotating}
\usepackage{color}
%\usepackage{upgreek}
%\usepackage{setspace}
%\usepackage{afterpage}

\newtheorem{thm}{Theorem}
\newtheorem{cor}[thm]{Corollary}
\newtheorem{lem}[thm]{Lemma}
\newtheorem{dfn}{Definition}
\newtheorem{pbm}{Problem}
\newtheorem{claim}{Claim}

\input{./macros}   % some pre-defined macros


\begin{document}

\title{Your Project Title}

\author{Student Name\\
        Department of Computer Science\\
        National Tsing Hua University\\
        Hsinchu, Taiwan\\
        account@cs.nthu.edu.tw \\ 
    % {\vspace{0.2cm} \large \bf Technical Report, September 2011} 
}


\date{}
\maketitle

\begin{abstract}

The abstract of your work should state the problem, why it is important, your
proposed solution, and the evaluation results. It should be concise, say
200-250 words.

\end{abstract}

% The body of the document: divided into multiple sections When the file grows,
% you may want to break sections into separate files. Check \input{...} command
% in Latex.

\section{Introduction} \label{sec:introduction}

Motivate your problem. Why we need a better solution. Then, clearly state what
problem we address in this paper. How we validate that our solution is much
better than state-of-the-art solutions in the literature. 

\section{Related Work} \label{sec:related}

This is just a sample on how to cite different publications, e.g., 
journal papers \cite{MEVS03}, conference papers \cite{SGD+02}, 
technical reports \cite{MKL+02}, books \cite{comerV1}, papers 
in edited books \cite{CP02}, and miscellaneous \cite{freePastry, nntp}. 

You will see in the refs.bib file the completed entries for the above 
citations. Please use the same format for similar publications  
(copy and paste entries, then change data). 

When you read a paper, you should summarize the most important issues/ideas
presented in the paper in your related work section.  You should also try to
criticize the work and find shortcomings, unrealistic assumptions, etc, and how
this work could potentially be extended (why your solution is better). 

\section{Problem Statement} \label{sec:problem}

Formally state the problem clearly. You probably need to develop some notations
and introduce system models in this section. 

\section{Proposed Solution} \label{sec:solution}

Here is the brilliant solutions that you came up with for the problem. 

\section{Evaluation} \label{sec:evaluation}

In this section, you demonstrate that your solution is cool and it outperforms
previous works. This is usually done through simulations, but real experimental
results are much more convincing. Having a small prototype often significantly
increase your chance to get into top conferences. 

\section{Conclusions and Future Work} \label{sec:conclusion}

What are the lessons that we should learn from this paper? 
What are the possible extensions of this work?

%%% The References (are kept in a separate file "refs.bib" in this example. 
\bibliographystyle{IEEEtran}   
\bibliography{./refs}    % refs.bib --> should contain all references

\end{document}


