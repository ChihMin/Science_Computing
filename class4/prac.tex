\documentclass{article}
\usepackage{amssymb, amsmath}
\begin{document}
  \begin{enumerate}
    \item APPLE
    \item Orange
    \begin{enumerate}
      \item Orange
      \item Apple
      \begin{enumerate}
        \item gg
        \item haha
        \begin{description}
          \item [Router] This is router:D
        \end{description}
      \end{enumerate}
    \end{enumerate}
    \item yaya
    \item $a         b$
    \item $width \times height$
  \end{enumerate}
  $\mathit{width} \times \mathit{height}$
  $\binom{\frac{n^2-1}{2}}{n+1}$
  \begin{equation}
    \displaystyle{\sum_{i=1}^{[\frac{n}{2}]}\binom{x^{i^2}_{i,i+1}}{[\frac{i+3}{3}]}\frac{\sqrt{\mu(i)^{\frac{3}{2}}(i^2-1)}}{\sqrt[3]{\rho(i)-2}+\sqrt[3]{\rho(i)-1}}}
  \end{equation}

  \begin{gather}
    x_1x_2 + y_1y_2 = 1000\\
    t_2t_3 + g_1g_2 = 234324 \nonumber\\
    x = 2123123
  \end{gather}

  \begin{multline}
    x + 2 + d + 4 + g +\\
    t+2+f+s+d+a+\\
    t + y + d +S +S +S-asd = 100000
  \end{multline}
  \begin{align*}
    f(x) &= 1 + 2 + 4 & g(x) &= 2 + 4 + 5  & g(x)f(x) &= i + h + j\\
    f(x) &= g + 3 + h & g(x) &= h + y + t  & t(x) &= j + eer+ 23
  \end{align*}

  \begin{equation*}
    \left|\left|
      \begin{matrix}
        a+b+c & a+C & g+c & sdfsdf\\
        t & yyyy +xxxx & sdfdsf & shi\\
        ss & \hdotsfor{2}  & tmd\\
      \end{matrix}
    \right|\right|
  \end{equation*}

  \begin{equation*}
    \left[
      \begin{array}{lccr}
        a+b+c & a+C & g+c & sdfsdf\\
        t & yyyy +xxxx & sdfdsf & shi\\
        ss & \hdotsfor{2}  & tmd\\
      \end{array}
    \right]
  \end{equation*}

  \begin{equation*}
    \begin{array}{r|lcr}
      & a+C & g+c & sdfsdf\\
      \hline
      t & yyyy +xxxx & sdfdsf & shi\\
      ss & 1  & 22  & tmd
    \end{array}
  \end{equation*}
  
  this is the first sequence 
  this is the first sequence \\
  this is the first sequence \\
  this is the first sequence \\
  \begin{table}[tbh]
  \caption{FLying Disk Distance (m)}
  \begin{center}
    \begin{tabular}{|l|r|r|r|}
      \hline
      & 1 & 2 & 3 \\
      \hline
      12 & 2 & 3 & 4 \\
      \hline
      ttt & 5 & f & 23 \\
      \hline
    \end{tabular}
  \end{center}
  \end{table}
\end{document}

