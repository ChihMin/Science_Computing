\documentclass{article}
\usepackage{amssymb, amsmath}
\usepackage{graphicx}
\usepackage{epstopdf}
\usepackage[utf8]{inputenc}
\usepackage{mdwlist}
\usepackage[left=1cm, right=2cm, top=1.5cm, bottom=1.2cm]{geometry}
\epstopdfsetup{outdir=./}
\begin{document}

  \begin{enumerate*}
    \item [1.]
    \begin{enumerate*}  
      \item [(a)] \text{}\\
      \includegraphics[width=0.7\textwidth]{$PWD/Q1_1.png}

	  \item [(b)] \text{}\\ 
      The (a) figure show that the value of y will be changed in large range(from positive infinity to negative infinity) between x-value -1 to -5. Therefore, we cannot calculate accurate answer for this formular and we cannot get a numeracal answer.
      
      \item [(c)] \text{}\\ 

    \end{enumerate*}
    
    \item [2.]
    \begin{enumerate*}
      \item [(a)]
      $\because$ p is prime, and p is coprime between all integers from 1 to p except number p \\
      $\therefore \ phi(p)\ is\ p-1$ 
      
      \item [(b)]
      Obervation: \\
      $\phi(11) \times \phi(13) = \phi(11 \times 13)$ \\
      In my opinion, all distinct prime numbers p and q can be done by this observation because p and q are different numbers and they are coprime.
      
      \item [(c)]
        \ 
        \begin{enumerate*}
          \item [(i)]
          Suppose we have distinct prime number p and q.
          \\

          \item [(ii)]
          $\because$ p is prime number, and q is also prime number \\
          $\therefore\ \phi(p) = p-1,\ \phi(q) = q-1,\ \phi(p)\phi(q)=(p-1)(q-1)$
          \\

          \item [(iii)]
          $\because$ \\
          We observe "number set" coprime between $p \times q$ in range $[1,\ p\times q]$ : \\
          $\left\{ 1,\ 2,\ 3,\ ,\cdots\cdots,\ p \times q \right\}\ -$
          $\{p \times 1,\ p \times 2,\ \cdots\cdots p\times q \}\ -$
          $\{q \times 1,\ q \times 2,\ \cdots\cdots q\times p \}\ +$
          $\{p \times q\}$ \\
          $\therefore$ \\ 
          The number of $PRIME\ SET\ =$ 
          $\phi(p \times q)=p \times q - p - q + 1$ 
          \\

          \item [(iv)]
          $\because(ii)\ is\ equal\ to\ (iii)$ \\
          $\therefore\ the\ equation\ is\ hold.$
           
        \end{enumerate*}
      \item [3.]
      \begin{enumerate*}
        \item [(a)]
        Use is\_prime() function to check whether a number is prime or not.\\
        Sample code : is\_prime(2**1279-1)
      \end{enumerate*}

    \end{enumerate*}
  \end{enumerate*}
\end{document}

